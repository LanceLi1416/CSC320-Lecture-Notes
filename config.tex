% ------------------
% Packages
% ------------------
\usepackage[english]{babel}
\usepackage[T1]{fontenc}
\usepackage[scaled]{helvet}
\usepackage[utf8]{inputenc}
\usepackage[letterpaper]{geometry}
\usepackage[dvipsnames]{xcolor}

\usepackage{algorithm}
\usepackage{algorithmicx}
\usepackage{algpseudocode}
\usepackage{amsfonts,amsmath,amsthm,amssymb}
\usepackage{array}
\usepackage{booktabs}
\usepackage{float}
\usepackage{graphicx}
\usepackage{latexsym}
\usepackage{mathtools}
\usepackage{sectsty}
\usepackage{setspace}
\usepackage{subcaption}
\usepackage{tabularx}
\usepackage{tikz}
% \usepackage{txfonts}
\usepackage{url}
\usepackage{wrapfig}
\usepackage{xparse}

% ------------------
% Shortcuts
% ------------------
\newcommand{\R}{\mathbb{R}}
\newcommand{\N}{\mathbb{N}}
\newcommand{\id}{\mathrm{id}}

\newcommand{\Frontier}{\texttt{Frontier}}

\newcommand{\bigo}[1]{\mathcal{O}\left(#1\right)}
\newcommand{\floor}[1]{\left\lfloor #1 \right\rfloor}


% ------------------
% Command Redefinitions
% ------------------
\renewcommand{\qed}{\hfill$\blacksquare$}

\makeatletter
\renewcommand*\env@matrix[1][*\c@MaxMatrixCols c]{%
    \hskip -\arraycolsep
    \let\@ifnextchar\new@ifnextchar
    \array{#1}}
\makeatother

\newcolumntype{Y}{>{\centering\arraybackslash}X}


% ------------------
% Colors
% ------------------
\definecolor{primary}{HTML}{207BA5}
\definecolor{greybg}{RGB}{249, 249, 249}

\definecolor{thmbg}{HTML}{F2F2F9}
\definecolor{lemmabg}{HTML}{FFFAF8}
\definecolor{lemmafr}{HTML}{983b0f}
\definecolor{propbg}{HTML}{f2fbfc}
\definecolor{propfr}{HTML}{191971}
\definecolor{myp}{RGB}{197, 92, 212}
\definecolor{grey17}{RGB}{17, 17, 17}
\definecolor{MyGrey}{HTML}{5B5B5B}

\definecolor{lightBlue}{rgb}{0.0, 0.64, 1.0}
\definecolor{lightRed}{rgb}{1.0, 0.50, 0.50}
\definecolor{darkGreen}{rgb}{0.31, 0.54, 0.30}
\definecolor{violet}{RGB}{186, 153, 242}
\definecolor{darkRed}{HTML}{b91231}

\definecolor{algo-kwd}{HTML}{e43c48}

% https://m2.material.io/design/color/the-color-system.html#tools-for-picking-colors
\definecolor{Red-50}         {HTML}{FFEBEE}
\definecolor{Red-100}        {HTML}{FFCDD2}
\definecolor{Red-200}        {HTML}{EF9A9A}
\definecolor{Red-300}        {HTML}{E57373}
\definecolor{Red-400}        {HTML}{EF5350}
\definecolor{Red-500}        {HTML}{F44336}
\definecolor{Red-600}        {HTML}{E53935}
\definecolor{Red-700}        {HTML}{D32F2F}
\definecolor{Red-800}        {HTML}{C62828}
\definecolor{Red-900}        {HTML}{B71C1C}
\definecolor{Red-A100}       {HTML}{FF8A80}
\definecolor{Red-A200}       {HTML}{FF5252}
\definecolor{Red-A400}       {HTML}{FF1744}
\definecolor{Red-A700}       {HTML}{D50000}

\definecolor{Pink-50}        {HTML}{FCE4EC}
\definecolor{Pink-100}       {HTML}{F8BBD0}
\definecolor{Pink-200}       {HTML}{F48FB1}
\definecolor{Pink-300}       {HTML}{F06292}
\definecolor{Pink-400}       {HTML}{EC407A}
\definecolor{Pink-500}       {HTML}{E91E63}
\definecolor{Pink-600}       {HTML}{D81B60}
\definecolor{Pink-700}       {HTML}{C2185B}
\definecolor{Pink-800}       {HTML}{AD1457}
\definecolor{Pink-900}       {HTML}{880E4F}
\definecolor{Pink-A100}      {HTML}{FF80AB}
\definecolor{Pink-A200}      {HTML}{FF4081}
\definecolor{Pink-A400}      {HTML}{F50057}
\definecolor{Pink-A700}      {HTML}{C51162}

\definecolor{Purple-50}      {HTML}{F3E5F5}
\definecolor{Purple-100}     {HTML}{E1BEE7}
\definecolor{Purple-200}     {HTML}{CE93D8}
\definecolor{Purple-300}     {HTML}{BA68C8}
\definecolor{Purple-400}     {HTML}{AB47BC}
\definecolor{Purple-500}     {HTML}{9C27B0}
\definecolor{Purple-600}     {HTML}{8E24AA}
\definecolor{Purple-700}     {HTML}{7B1FA2}
\definecolor{Purple-800}     {HTML}{6A1B9A}
\definecolor{Purple-900}     {HTML}{4A148C}
\definecolor{Purple-A100}    {HTML}{EA80FC}
\definecolor{Purple-A200}    {HTML}{E040FB}
\definecolor{Purple-A400}    {HTML}{D500F9}
\definecolor{Purple-A700}    {HTML}{AA00FF}

\definecolor{Indigo-50}      {HTML}{E8EAF6}
\definecolor{Indigo-100}     {HTML}{C5CAE9}
\definecolor{Indigo-200}     {HTML}{9FA8DA}
\definecolor{Indigo-300}     {HTML}{7986CB}
\definecolor{Indigo-400}     {HTML}{5C6BC0}
\definecolor{Indigo-500}     {HTML}{3F51B5}
\definecolor{Indigo-600}     {HTML}{3949AB}
\definecolor{Indigo-700}     {HTML}{303F9F}
\definecolor{Indigo-800}     {HTML}{283593}
\definecolor{Indigo-900}     {HTML}{1A237E}
\definecolor{Indigo-A100}    {HTML}{8C9EFF}
\definecolor{Indigo-A200}    {HTML}{536DFE}
\definecolor{Indigo-A400}    {HTML}{3D5AFE}
\definecolor{Indigo-A700}    {HTML}{304FFE}

\definecolor{Blue-50}        {HTML}{E3F2FD}
\definecolor{Blue-100}       {HTML}{BBDEFB}
\definecolor{Blue-200}       {HTML}{90CAF9}
\definecolor{Blue-300}       {HTML}{64B5F6}
\definecolor{Blue-400}       {HTML}{42A5F5}
\definecolor{Blue-500}       {HTML}{2196F3}
\definecolor{Blue-600}       {HTML}{1E88E5}
\definecolor{Blue-700}       {HTML}{1976D2}
\definecolor{Blue-800}       {HTML}{1565C0}
\definecolor{Blue-900}       {HTML}{0D47A1}
\definecolor{Blue-A100}      {HTML}{82B1FF}
\definecolor{Blue-A200}      {HTML}{448AFF}
\definecolor{Blue-A400}      {HTML}{2979FF}
\definecolor{Blue-A700}      {HTML}{2962FF}

\definecolor{Light-Blue-50}  {HTML}{E1F5FE}
\definecolor{Light-Blue-100} {HTML}{B3E5FC}
\definecolor{Light-Blue-200} {HTML}{81D4FA}
\definecolor{Light-Blue-300} {HTML}{4FC3F7}
\definecolor{Light-Blue-400} {HTML}{29B6F6}
\definecolor{Light-Blue-500} {HTML}{03A9F4}
\definecolor{Light-Blue-600} {HTML}{039BE5}
\definecolor{Light-Blue-700} {HTML}{0288D1}
\definecolor{Light-Blue-800} {HTML}{0277BD}
\definecolor{Light-Blue-900} {HTML}{01579B}
\definecolor{Light-Blue-A100}{HTML}{80D8FF}
\definecolor{Light-Blue-A200}{HTML}{40C4FF}
\definecolor{Light-Blue-A400}{HTML}{00B0FF}
\definecolor{Light-Blue-A700}{HTML}{0091EA}

\definecolor{Orange-50}      {HTML}{FFF3E0}
\definecolor{Orange-100}     {HTML}{FFE0B2}
\definecolor{Orange-200}     {HTML}{FFCC80}
\definecolor{Orange-300}     {HTML}{FFB74D}
\definecolor{Orange-400}     {HTML}{FFA726}
\definecolor{Orange-500}     {HTML}{FF9800}
\definecolor{Orange-600}     {HTML}{FB8C00}
\definecolor{Orange-700}     {HTML}{F57C00}
\definecolor{Orange-800}     {HTML}{EF6C00}
\definecolor{Orange-900}     {HTML}{E65100}
\definecolor{Orange-A100}    {HTML}{FFD180}
\definecolor{Orange-A200}    {HTML}{FFAB40}
\definecolor{Orange-A400}    {HTML}{FF9100}
\definecolor{Orange-A700}    {HTML}{FF6D00}

\definecolor{Gray-50} {HTML}{FAFAFA}
\definecolor{Gray-100}{HTML}{F5F5F5}
\definecolor{Gray-200}{HTML}{EEEEEE}
\definecolor{Gray-300}{HTML}{E0E0E0}
\definecolor{Gray-400}{HTML}{BDBDBD}
\definecolor{Gray-500}{HTML}{9E9E9E}
\definecolor{Gray-600}{HTML}{757575}
\definecolor{Gray-700}{HTML}{616161}
\definecolor{Gray-800}{HTML}{424242}
\definecolor{Gray-900}{HTML}{212121}

%----------------
%	Text Styles
%----------------
% \DeclareTextFontCommand{\term}{\color{orange}\bfseries}
% \DeclareTextFontCommand{\bred}{\color{darkRed}\bfseries}
% \DeclareTextFontCommand{\itblue}{\color{lightBlue}\itshape}

\DeclareTextFontCommand{\term}{\color{Orange-900}\bfseries}

\DeclareTextFontCommand{\bold}{\bfseries}
\DeclareTextFontCommand{\italic}{\itshape}

\renewcommand{\textbf}[1]{\textcolor{Red-800}{\bfseries #1}}
\renewcommand{\textit}[1]{\textcolor{Light-Blue-400}{\itshape #1}}

\DeclareTextFontCommand{\vector}{\bfseries\itshape}

% ------------------
%   URL Color
% ------------------
\usepackage[colorlinks=true]{hyperref}
\hypersetup{
    colorlinks=true,
    linkcolor=black,
    filecolor=magenta,
    urlcolor=blue,
}


% ------------------
% Tikz Externalize
% ------------------
\usetikzlibrary{external}
\tikzexternalize[prefix=tikz/]
\tikzexternaldisable
\BeforeBeginEnvironment[label]{tikzpicture}{\tikzexternalenable}
\AfterEndEnvironment[label]{tikzpicture}{\tikzexternaldisable}

% ------------------
% Algorithms
% ------------------
% Change color of keywords
\algrenewcommand\algorithmicfunction{\textcolor{algo-kwd}{\textbf{function}}}
\algrenewcommand\algorithmicreturn{\textcolor{algo-kwd}{\textbf{return}}}

\algrenewcommand\algorithmicfor{\textcolor{algo-kwd}{\textbf{for}}}
\algrenewcommand\algorithmicwhile{\textcolor{algo-kwd}{\textbf{while}}}
\algrenewcommand\algorithmicdo{\textcolor{algo-kwd}{\textbf{do}}}

\algrenewcommand\algorithmicif{\textcolor{algo-kwd}{\textbf{if}}}
\algrenewcommand\algorithmicthen{\textcolor{algo-kwd}{\textbf{then}}}
\algrenewcommand\algorithmicelse{\textcolor{algo-kwd}{\textbf{else}}}

\algrenewcommand\Call[2]{\textproc{\color{primary}\textsc{#1}}(#2)}

% No end if, end for, etc.
\algtext*{EndIf}
\algtext*{EndFor}
\algtext*{EndWhile}
\algtext*{EndFunction}

% ------------------
% Boxes
% ------------------
\usepackage[most]{tcolorbox}

\newcommand\fancybox[3]{%
    \tcbset{
        mybox/.style={
                enhanced,
                boxsep=0mm,
                opacityfill=0,
                overlay={
                        \coordinate (X) at ([xshift=-1mm, yshift=-1.5mm]frame.north west);
                        \node[align=right, text=#1, text width=2.5cm, anchor=north east] at (X) {\bf#2};
                        \draw[line width=0.5mm, color=#1] (frame.north west) -- (frame.south west);
                    }
            }
    }
    \begin{tcolorbox}[mybox]
        #3
    \end{tcolorbox}
}

\tcbuselibrary{theorems,skins,hooks}
\NewDocumentCommand\thmbox{m O{\Large #1} O{greybg} O{primary} O{number within=section}}
{
    \newtcbtheorem[#5]{#1}{\large #2}
    {%
        enhanced,
        breakable,
        colback = #3,
        frame hidden,
        boxrule = 0sp,
        borderline west = {2pt}{0pt}{#4},
        sharp corners,
        detach title,
        before upper = \tcbtitle\par\smallskip,
        coltitle = #4,
        fonttitle = \bfseries,
        %description font = \mdseries,
        separator sign none,
        segmentation style={solid, #4}
    }
    {th}
}

\thmbox{Corollary}[Corollary][myp!10][myp!85!black]
\thmbox{Lemma}[Lemma][lemmabg][lemmafr]
\thmbox{Propo}[Proposition][propbg][propfr]
\thmbox{Defi}[Definition][primary!12][primary]
\thmbox{Notation}[Notation][white][grey17][no counter]
\thmbox{Theorem}[Theorem][primary!12][primary]
\thmbox{Remark}[Remark][grey17!10][grey17][no counter]

% ------------------
% Environments
% ------------------
\newenvironment{corollary}[1][]   {\begin{Corollary}{#1}{}}                               {\end{Corollary}}
\newenvironment{definition}[1][]  {\begin{Defi}{#1}{}}                                    {\end{Defi}}
\newenvironment{lemma}[1][]       {\begin{Lemma}{#1}{}}                                   {\end{Lemma}}
\newenvironment{proposition}[1][] {\begin{Propo}{#1}{}}                                   {\end{Propo}}
\newenvironment{remark}[1][]      {\begin{Remark}{#1}{}}                                  {\end{Remark}}
\newenvironment{theorem}[1][]     {\begin{Theorem}{#1}{}}                                 {\end{Theorem}}

\newenvironment{rtheorem}[2][]    {\begin{Theorem}{#1}{#2}}                               {\end{Theorem}}

\newenvironment{solution}         {\color{primary}\textbf{Solution:}}                     {}

\theoremstyle{definition}
\newtheorem*{exam}{\color{primary}Example}
% \newcommand{\example}[1]{\begin{exam}#1\end{exam}}
% \newenvironment{example}          {\begin{exam}} {\begin{flushright}${\color{primary}\diamondsuit}$\end{flushright} \end{exam}}
\newenvironment{example}          {\begin{exam}} {\hfill${\color{primary}\diamondsuit}$\end{exam}}

\theoremstyle{definition}
\newtheorem*{clm}{\color{MyGrey}Claim}
\newenvironment{claim}            {\begin{clm}} {\end{clm}}

% ------------------
% Lists
% ------------------
\usepackage{enumitem}

\newcommand{\cnumero}[2]{
    \tikz[baseline=(myanchor.base)]
    \node[minimum size=0.2cm,circle,
        inner sep=1pt,draw, #2,thick,fill=#2](myanchor)
    {\color{white}\bfseries\fontsize{8}{8}#1};}

\newcommand*{\itembolasazules}[1]{\protect\cnumero{#1}{primary}}

% \newenvironment{listo} {\begin{enumerate}[label=\itembolasazules{\arabic*}]} {\end{enumerate}}
% \newenvironment{listu} {\begin{itemize}  [label=$\color{primary} \bullet$]}  {\end{itemize}}

\setlist[itemize]{label=$\color{primary} \bullet$}
\setlist[enumerate]{label=\itembolasazules{\arabic*}}

% ------------------
% Table of Contents
% ------------------
\usepackage{blindtext}
\usepackage{framed}
\usepackage{titletoc}
\usepackage{etoolbox}

\patchcmd{\tableofcontents}{\contentsname}{\contentsname}{}{}

\renewenvironment{leftbar}
{\def\FrameCommand{\hspace{6em}%
        {\color{primary}\vrule width 2pt depth 6pt}\hspace{1em}}%
    \MakeFramed{\parshape 1 0cm \dimexpr\textwidth-6em\relax\FrameRestore}\vskip2pt%
}
{\endMakeFramed}

\titlecontents{chapter}[0em]
{\vspace*{2\baselineskip}}
{\parbox{4.5em}{%
        \hfill\Huge\bfseries\color{primary}\thecontentslabel}%
    \vspace*{-2.3\baselineskip}\leftbar\textbf{\color{primary}\small\chaptername~\thecontentslabel}\\
}{}{\endleftbar}

\titlecontents{section}[8.4em]
{\contentslabel{3em}}{}{}
{\hspace{0.5em}\nobreak\itshape\color{primary}\contentspage}

\titlecontents{subsection}[11.4em]
{\contentslabel{3em}}{}{}
{\hspace{0.5em}\nobreak\itshape\color{primary}\contentspage}

% ------------------
% Chapters
% ------------------
\newtcolorbox{titlecolorbox}[1]{ %the box around chapter
    coltext=white,
    colframe=primary,
    colback=primary,
    boxrule=0pt,
    arc=0pt,
    notitle,
    width=4.8em,
    height=2.4ex,
    before=\hfill
}

\usepackage[explicit]{titlesec}

\makeatletter
\let\old@rule\@rule
\def\@rule[#1]#2#3{\textcolor{primary}{\old@rule[#1]{#2}{#3}}}
\makeatother

\titleformat{\chapter}[display]
{\Huge}
{}
{0pt}
{\begin{titlecolorbox}{}
        {\large\MakeUppercase{\bf\chaptername}}
    \end{titlecolorbox}
    \vspace*{-3.19ex}\noindent\rule{\textwidth}{0.4pt}
    \parbox[b]{\dimexpr\textwidth-4.8em\relax}{\raggedright\MakeUppercase{#1}}{\hfill\fontsize{70}{60}\selectfont{\color{primary}\thechapter}}
}
[]

\titleformat{name=\chapter,numberless}[display]
{\Huge}
{}
{0pt}
{
    \vspace*{-3.19ex}\noindent\rule{\textwidth}{0.4pt}
    \parbox[b]{\dimexpr\textwidth-4.8em\relax}{\raggedright\MakeUppercase{#1}}
}
[]

% ------------------
% Sections
% ------------------
\titleformat{\section}[hang]{\Large\bfseries}%
{\rlap{\color{primary}\rule[-6pt]{\textwidth}{0.4pt}}\colorbox{primary}{%
        \raisebox{0pt}[13pt][3pt]{ \makebox[60pt]{% height, width
                \selectfont\color{white}{\thesection}}
        }}}%
{15pt}%
{ \color{primary}#1
    %
}
\titlespacing*{\section}{0pt}{3mm}{5mm}
% ------------------
% Subsections
% ------------------
\subsectionfont{\Large\color{primary}}

% ------------------
% Bibliography and Index
% ------------------
\usepackage{csquotes}
\usepackage[
    style=alphabetic, 
    citestyle=numeric,
    sorting=nyt,
    sortcites=true,
    autopunct=true,
    autolang=hyphen,
    hyperref=true,
    abbreviate=false,
    backref=true,
    backend=biber,
    defernumbers=true
]{biblatex}
\addbibresource{./bibliography.bib} % BibTeX bibliography file
\defbibheading{bibempty}{}

\usepackage{calc} % For simpler calculation - used for spacing the index letter headings correctly
\usepackage{makeidx} % Required to make an index
\makeindex % Tells LaTeX to create the files required for indexing

% ------------------
% Title page
% ------------------
\usetikzlibrary{calc}
\usetikzlibrary{shapes.geometric}
\usepackage{anyfontsize}
\newcommand{\frontpage}[3]{
    \tikzset{external/export next=false}
    \begin{tikzpicture}[remember picture, overlay]
        % Background
        \fill[primary] (current page.south west) rectangle (current page.north east);

        \foreach \i in {2.5,...,22} {
            \node[rounded corners,primary!60,draw,regular polygon,regular polygon sides=6, minimum size=\i cm,ultra thick] at ($(current page.west)+(2.5,-5)$) {} ;
        }

        % Background Polygon
        \foreach \i in {0.5,...,22} {
            \node[rounded corners,primary!60,draw,regular polygon,regular polygon sides=6, minimum size=\i cm,ultra thick] at ($(current page.north west)+(2.5,0)$) {} ;
        }

        \foreach \i in {0.5,...,22} {
            \node[rounded corners,primary!90,draw,regular polygon,regular polygon sides=6, minimum size=\i cm,ultra thick] at ($(current page.north east)+(0,-9.5)$) {} ;
        }

        \foreach \i in {21,...,6} {
            \node[primary!85,rounded corners,draw,regular polygon,regular polygon sides=6, minimum size=\i cm,ultra thick] at ($(current page.south east)+(-0.2,-0.45)$) {} ;
        }

        % Title
        \node[left,primary!5,minimum width=0.625*\paperwidth,minimum height=3cm, rounded corners] at ($(current page.north east)+(0,-9.5)$) {
            {\fontsize{25}{30} \selectfont \bfseries #1}
        };

        % Subtitle
        \node[left,primary!10,minimum width=0.625*\paperwidth,minimum height=2cm, rounded corners] at ($(current page.north east)+(0,-11)$) {
            {\huge \textit{#2}}
        };

        % Author
        \node[left,primary!5,minimum width=0.625*\paperwidth,minimum height=2cm, rounded corners] at ($(current page.north east)+(0,-13)$) {
            {\Large \textsc{#3}}
        };

        % Year
        \node[rounded corners,fill=primary!70,text =primary!5,regular polygon,regular polygon sides=6, minimum size=2.5 cm,inner sep=0,ultra thick] at ($(current page.west)+(2.5,-5)$) {\LARGE \bfseries \the\year{}};
    \end{tikzpicture}
}
